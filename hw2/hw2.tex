\documentclass{article}
\usepackage{fancyhdr}
\usepackage{lipsum}  
\usepackage{listings} 
\usepackage{xcolor}   
\usepackage{amsmath}
\usepackage{enumitem}

% Define macros for title and author
\newcommand{\thetitle}{MATH 410 502 \\ Homework 2}
\newcommand{\theauthor}{Keegan Smith}

\title{\thetitle}
\author{\theauthor}

\pagestyle{fancy}
\fancyhf{}  % Clear all header and footer fields
\fancyhead[L]{\nouppercase{\rightmark}}
\fancyhead[C]{\thetitle}  % Title in the center
\fancyhead[R]{\theauthor}  % Your name on the right
\begin{document}
\maketitle
\section*{Problem 1}
Initially our memory space looks like:\\
16 byte free list: [0, ..., 15] \\
8 byte free list: [] \\
4 byte free list: [] \\
2 byte free list: [] \\
1 byte free list: [] \\
after 1 byte is requested: \\
16 byte free list: [] \\
8 byte free list: [8,..., 15] \\
4 byte free list: [4,..., 7] \\
2 byte free list: [2, 3] \\
1 byte free list: [1] \\
after 2 byte requested (so now 3 bytes in total requested):\\
16 byte free list: [] \\
8 byte free list: [8,..,15] \\
4 byte free list: [4,..., 7] \\
2 byte free list: [] \\
1 byte free list: [1] \\
after 4 byte requested (so now 7 bytes in total requested):\\
16 byte free list: [] \\
8 byte free list: [8,..,15] \\
4 byte free list: [] \\
2 byte free list: [] \\
1 byte free list: [1] \\
after 2 byte requested (so now 9 bytes in total requested):\\
16 byte free list: [] \\
8 byte free list: [] \\
4 byte free list: [12,...,15] \\
2 byte free list: [10,11] \\
1 byte free list: [1] \\
\section*{Problem 2}
\begin{enumerate}[label=\alph*)]
\item There are $\frac{2^{36}}{2^{13}} = 2^{23}$ pages in the virtual address space. \\
\item If each page table entry is 4 bytes = $2^5 = 32$ bits, then there must be $2^{32}$ physical pages. Thus the total addressable memory should be that times the size of a page: $2^{32} \cdot 2^{13} = 2^{45}$ bytes. \\
\item I would use a 1 level page table because an 8GB process would be using most of the memory anyways. A 2 or 3 level page table would only serve to obfuscate things as you would still have to allocate memory for pretty much all of the 2nd and 3rd level pages. \\
\item For a single page table, the number of page table entries would be $\frac{2^33}{2^13} = 2^{20}$.
\end{enumerate}
\section*{Problem 3}
\begin{enumerate}[label=\alph*)]
\item The page size would be $2^8$ bytes\\
\item A process that has $2^{18}$ bytes would be using $\frac{2^{18}}{2^8} = 2^{10}$ pages. This means that the number of allocated page table entries in the third level table must be $2^{10}$. \\Each third level page table has $2^6$ entries, so there must have been $\frac{2^{10}}{2^6} = 2^4$ 2nd level page entries allocated (because $2^4$ third level pages were allocated).\\ The size of a level 2 page table is $2^8$ entries, since only $2^4$ entries were needed, we only allocated one level 2 page table.\\ This means we only needed one page entry in the level 1 page table, and thus we only allocated one level one page table. So in total, we allocated 3 level 3 tables, one level 2 table, and one level one table. \\The total number of pages is $\frac{2^{32}}{2^8} = 2^{24}$, so the size of a page entry is 24 bits or 3 bytes. \\
So in total, the page table size was $2^6 \cdot 2^4 \cdot 3 + 2^8\cdot 3 + 2^{10} \cdot 3$ bytes, and wasted $(2^8 - 3 + 2^{10} - 1) \cdot 3$ bytes due to internal fragmentation\\ 
\item For the code segment, we are allocating a total of $2^{14} \cdot 3$ bytes which is $\frac{2^{14} \cdot 3 }{2^8} = 2^6 \cdot 3$ pages. Thus we need $2^6 \cdot 3$ entries in our L3 page table, L3 has $2^6$ entries per table, so we allocated 3 tables. Since we allocated 3 tables in L3, we must've used 3 entries in L2 page table, the number of entries in an L2 table is $2^8$ so we only needed to allocate 1 L2 page table. So of course we only allocated 1 L1 page table. So for the code segment's page tables we consumed: $(2^6\cdot 3 + 2^8 + 2^10) \cdot 3$ bytes. Memory which was not used (in the L1 and L2 page tables) is $(2^8 - 3 + 2^{10} - 1) \cdot3 $ (we only used 3 entries in the L2 table and only one entry in L1). \\
For the data segment: $600K = 75 \cdot 2^{15}$ bytes $= \frac{75 \cdot 2^{15}}{2^8}$ pages $= 75 \cdot 2^7$ pages. So we will need $\frac{75 \cdot 2^7}{2^6} = 75 \cdot 2$ L3 page tables. So we will need $\lceil \frac{75 \cdot 2}{2^8} \rceil = 1$ L2 page (which is wasting $(2^8 - 150) \cdot 3 = 318$ bytes). We are only using 1 L1 entry so 1 L1 page table, and so we are wasting $(2^{10} -1) \cdot 3$ bytes. \\
In total for the data segment, we allocated: $(75 \cdot 2 \cdot 2^6 + 2^8 + 2^{10}) \cdot 3$ bytes. \\
We wasted $318 + (2^{10} -1) \cdot 3$ bytes. \\
For the stack segment, I already did this in b). We allocated a total of $2^6 \cdot 2^4 \cdot 3 + 2^8\cdot 3 + 2^{10} \cdot 3$ and wasted $(2^8 - 3 + 2^{10} - 1) \cdot 3$ bytes.  \\
So in total we allocated $(2^6\cdot 3 + 2^8 + 2^{10}) \cdot 3 + (75 \cdot 2 \cdot 2^6 + 2^8 + 2^{10}) \cdot 3 + 2^6 \cdot 2^4 \cdot 3 + 2^8\cdot 3 + 2^{10} \cdot 3$ for page tables and wasted $(2^8 - 3 + 2^{10} - 1) \cdot 3 + 318 + (2^{10} -1) \cdot 3 + (2^8 - 3 + 2^{10} - 1) \cdot3$ bytes due to internal fragmentation. \\
\item The size of the page should just be the offset, $2^{12}$ bytes since the protection bits should be accounted for in the page table entries. 
\end{enumerate}
\end{document}